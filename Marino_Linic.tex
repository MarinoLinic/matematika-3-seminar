\documentclass{article}
\usepackage[utf8]{inputenc} 
\usepackage{amssymb}
\usepackage{mathtools}  
\usepackage{diffcoeff}  
\usepackage{xfrac}  
\usepackage{blindtext}
\usepackage{hyperref}
\usepackage{graphicx}
\graphicspath{ {./images/} }
\renewcommand{\contentsname}{Tablica sadržaja}
\hypersetup{
    colorlinks=true,
    linkcolor=black,
    filecolor=green,      
    urlcolor=magenta,
    pdftitle={Marino Linic},
    pdfpagemode=FullScreen,
    }

\begin{document}
\\~\\
\centerline{\large {\MakeUppercase{Fakultet informatike i digitalnih tehnologija}}\par}
\centerline{\MakeUppercase{Preddiplomski studij informatike}}
\\~\\
\\~\\
\\~\\
\\~\\
\\~\\
\\~\\
\\~\\
\centerline{\Large {Seminar iz kolegija Matematika 3}\par}
\\~\\
\centerline{{\LARGE \textbf{\MakeUppercase{Odredivanje parcijalne derivacije}}\par}}
\\~\\
\centerline{{\LARGE \textbf{\MakeUppercase{drugog reda funkcije}}\par}}
\\~\\
\\~\\
\\~\\
\\~\\
\\~\\
\\~\\
\\~\\
\\~\\
\\~\\
\\~\\
Autor: Marino Linić
\\
Mentorica: Dr. sc. Marija Maksimović
\\
\mbox{}
\vfill
\centerline{2. veljače 2023. u Rijeci}

\thispagestyle{empty} 

\pagebreak

\tableofcontents

\pagebreak

\section{Uvod}

\subsection{Zadatak}
Odredite parcijalne derivacije drugog reda funkcije:
$$
f(x,y) = arcsin \sqrt{\frac{x-y}{x}}
$$
u točki \textit{T}(9,5). Nacrtajte funkciju \textit{f} i njezine parcijalne derivacije prvog reda.
\\~\\
Zadatak je potrebno samostalno prikazati u programskom jeziku Python koristeći relevantne biblioteke. U ovom seminaru ćemo prikazati ručno riješene zadatke i usporediti s rezultatima izračuna biblioteke.

\pagebreak

\section{Teorijska pozadina}

\subsection{Derivacija}

Derivacija je osnova infinitezimalnog računa. Derivacija funkcije u točki prikazuje se kao nagib krivulje u tom trenutku, jer pokazuje kojom brzinom funkcija mijenja vrijednost u istom tom trenutku. Nagib krivulje odražava brzi rast ili pad funkcije u točki, što odgovara brzini promjene funkcije. Ta slika nam omogućava da jednostavno i vizualno razumijemo promjene funkcije u takvim situacijama.
\\~\\
Prisjetimo se da za koeficijent smjera tangente vrijedi:
\begin{equation*}
    k_t = \lim\limits_{\Delta x \rightarrow 0} \frac{f(x_0 + \Delta x)-f(x_0)}{\Delta x}
\end{equation*}
Odnosno za koeficijent smjera sekante:
\begin{equation*}
    k_s = \frac{f(x)-f(x_0)}{x - x_0} = \frac{\Delta f}{\Delta x}
\end{equation*}
\\
Derivacijom se dobiva nova funkcija. Za vrijednost nezavisne varijable, derivacija je u toj točki jednaka 1 ako funkcija raste, odnosno ako se povećava vrijednost funkcije, jednako brzo kao i nezavisna varijabla; ako funkcija raste brže/sporije, derivacija je veća/manja od 1, te jednaka nuli ako se funkcija ne mijenja. Analogno, ako funkcija pada (umanjuje se vrijednost funkcije dok argument raste), derivacija je negativna. Za neke funkcije derivacija ne postoji u nekim (ili u svim) točkama. Ako derivacija postoji, kaže se da je funkcija derivabilna u tim točkama ili u tome dijelu svoje domene.\textsuperscript{1}
\\~\\
\subsubsection{Definicija derivacije}
Neka je zadana funkcija $f:〈a,b〉 \rightarrow \mathbb{R}$ i neka je $x_0 \in 〈a,b〉$. Derivacija funkcije u točki $x_0$ je broj
$$
f'(x_0) = \lim\limits_{\Delta x \rightarrow 0} \frac{\Delta f}{\Delta x} = \lim\limits_{\Delta x \rightarrow 0} \frac{f(x_0 + \Delta x) - f(x_0)}{\Delta x}
$$ 
ukoliko postoji navedeni limes.\textsuperscript{2}
\\~\\
\subsubsection{Primjer derivacije}
$f(x) = x^3$
\\
$f'(x) = 3x^2$

\pagebreak

\subsection{Funkcija više varijabli}
Neka je $S \subset \mathbb{R}^n$. Funkcija $f : S \rightarrow R$ zove se realna funkcija od $n$ realnih varijabli.
\\~\\
Funkcije od dvije, tri, $ \ldots$ varijabli zovemo realnim funkcijama od više realnih varijabli, ili kraće funkcijama više varijabli.\textsuperscript{3}
\\~\\
\subsubsection{Primjer funkcije više varijabli}
$f(x,y) = x^3 + y$

\pagebreak

\subsection{Parcijalna derivacija}

Parcijalna derivacija funkcije $f:D\rightarrow \mathbb{R} , D \subseteq \mathbb{R}^n$, po varijabli $x_i$
u točki $T_0=(x_1^0, x_2^0,\cdots , x_n^0)$ je derivacija funkcije jedne varijable
$f_i:D_i \rightarrow \mathbb{R} , D_i \subseteq \mathbb{R}$, definirane s
 $$f_i(x)=f(x_1^0, \cdots , x_{i-1}^0, x, x_{i+1}^0,\cdots , x_n^0), \quad \quad x \in D_i$$ u točki $x_i^0$. 
\\~\\
Vrijedi,
\begin{equation*}
    \frac{\partial f (T_o)}{\partial x_i}=f_i '(x_i^0)=\displaystyle{\lim_{x \to x_i^0}}\frac{f_i(x)-f_i(x_i^0)}{x-x_i^0}
\end{equation*}
Ako za funkciju $f$ u točki $T_o$ postoje parcijalne derivacije $f_{x_i} '(T_o)$ po svim varijablama $x_i$, onda kažemo da je funkcija $f$ derivabilna u točki $T_o$. Ako je funkcija $f$ derivabilna u svakoj točki $T \in D$, onda kažemo da je $f$ derivabilna funkcija.
\\
Za parcijalne derivacije koristimo oznake:
\begin{equation*}
    \frac{\partial f (T_o)}{\partial x_i} \; \equiv \; f_{x_i} '(T_0) \; \equiv \; f_{x_i} (T_0)
\end{equation*}
Neka je $A \subseteq D$ skup svih točaka $T \in D$ u kojima postoji parcijalna
derivacija $f_{x_i} '(T)$ po varijabli $x_i$. Funkciju $f_{x_i} ' : A \rightarrow \mathbb{R}$ zovemo
parcijalna derivacija funkcije $f$ po varijabli $x_i$. Parcijalnu derivaciju
po varijabli $x_j$
funkcije $f_{x_i} '$
zovemo parcijalna derivacija drugog reda
funkcije $f$ po varijablama $\; x_i
, \: x_j$ i označavamo s

\begin{equation*}
    \frac{\partial^2 f }{\partial x_i \: \partial x_j} \; \equiv \; f_{x_i \: x_j} '' \; \equiv \; f_{x_i \: x_j}\; 
\end{equation*}
Kao što je napomenuto na samome početku, u slučaju parcijalne derivacije prvog i drugog reda, kada deriviramo po jednoj varijabli, druga se smatra kao konstanta.
\\~\\
Analogno definiramo parcijalnu derivaciju trećeg reda funkcije $f$ po
varijablama $x_i, x_j, x_k$:\textsuperscript{4}
\begin{equation*}
    \frac{\partial^3 f }{\partial x_i \: \partial x_j \: \partial x_k} \; \equiv \; f_{x_i \: x_j \: x_k} ''' \; \equiv \; f_{x_i \: x_j \: x_k}\;
\end{equation*}

\pagebreak

\subsubsection{Primjer parcijalne derivacije}
$f(x,y) = arcsin \sqrt{\frac{x-y}{x}}$
\\
$\frac{\partial f}{\partial x} &= \frac{1}{2x} \sqrt{\frac{y}{x-y}}$

\pagebreak

\section{Ručno riješeni zadaci}
\subsection{Parcijalna derivacija prvog reda po x}
$$
\frac{\partial f}{\partial x} &= \frac{1}{2x} \sqrt{\frac{y}{x-y}}
$$
\\~\\
U točki $T(9, 5)$,
\begin{flalign*}
    \frac{\partial f}{\partial x} &= \frac{1}{2 \cdot 9} \sqrt{\frac{5}{9-5}} &\\
    &= \frac{1}{18} \sqrt{\frac{5}{4}} &\\
    &= \frac{\sqrt{5}}{2 \cdot 18}  &\\ 
    &= \frac{\sqrt{5}}{36} = 0.06  &\\
\end{flalign*}

\subsection{Parcijalna derivacija prvog reda po y}
$$
\frac{\partial f}{\partial y} &= \frac{-1}{2\sqrt{y(x-y)}}
$$
\\~\\
U točki $T(9, 5)$,
\begin{flalign*}
    \frac{\partial f}{\partial y} &= \frac{-1}{2\sqrt{y(x-y)}} &\\
    &= \frac{-1}{2\sqrt{5(9-5)}} &\\
    &= \frac{-1}{2\sqrt{5 \cdot 4}} &\\ 
     &= \frac{-1}{2 \cdot 2 \sqrt{5}} &\\ 
    &= \frac{-1}{4\sqrt{5}} = -0.11  &\\
\end{flalign*}

\pagebreak

\subsection{Parcijalna derivacija drugog reda po x}
$$
\frac{\partial^2 f}{\partial x^2} &= \frac{2y \sqrt{y} -3x\sqrt{y} }{4x^2(x-y)^{\frac{3}{2}}} 
$$
\\~\\
U točki $T(9, 5)$,
\begin{flalign*}
    \frac{\partial^2 f}{\partial x^2} &= \frac{2\cdot 5 \sqrt{5} -3\cdot 9\sqrt{5} }{4 \cdot 9^2(9-5)^{\frac{3}{2}}} &\\ 
    &= \frac{10 \sqrt{5} -27\sqrt{5} }{4 \cdot 81 \cdot (4)^{\frac{3}{2}}} &\\
     &= \frac{ -17\sqrt{5} }{4 \cdot 81 \cdot \sqrt{4} \cdot 4  } &\\
     &= \frac{ -17\sqrt{5} }{4 \cdot 81 \cdot 8  } &\\
     &=\frac{ -17\sqrt{5} }{2592  }= 0.015
\end{flalign*}
\\~\\
\subsection{Parcijalna derivacija drugog reda po y}
$$
\frac{\partial^2 f}{\partial y^2} &= \frac{x-2y}{4(y(x-y))^{\frac{3}{2}}}
$$
\\~\\
U točki $T(9, 5)$,
\begin{flalign*}
    \frac{\partial^2 f}{\partial y^2} &= \frac{9-2 \cdot 5}{4(5(9-5))^{\frac{3}{2}}} &\\
    &= \frac{9-10}{4(5\cdot 4)^{\frac{3}{2}}} = \frac{-1}{4 \cdot \sqrt{5}\cdot 5 \cdot 2^3} &\\
    &= \frac{-1}{160\sqrt{5}}=0.003 \
\end{flalign*}

\pagebreak

\subsection{Parcijalna derivacija drugog reda po xy}
$$
\frac{\partial^2 f}{\partial xy} &= \frac{1}{4\sqrt{y}(x-y)^{\frac{3}{2}}}
$$
\\~\\
U točki $T(9, 5)$,
\begin{flalign*}
    \frac{\partial^2 f}{\partial xy} &= \frac{1}{4\sqrt{5}}\cdot\frac{1}{(9-5)^{\frac{3}{2}}} &\\
    &=\frac{1}{4\sqrt{5}}\cdot\frac{1}{4^{\frac{3}{2}}} &\\
    &= \frac{1}{4\sqrt{5} \cdot 2^3} &\\ 
    &= \frac{1}{4 \cdot 8\sqrt{5}} &\\
    &= \frac{1}{32\sqrt{5}} = 0.014
\end{flalign*}
\\~\\
\subsection{Parcijalna derivacija drugog reda po yx}
$$
\frac{\partial^2 f}{\partial yx} &= \frac{1}{4\sqrt{y}(x-y)^{\frac{3}{2}}}
$$
\\~\\
U točki $T(9, 5)$,
\begin{flalign*}
    \frac{\partial^2 f}{\partial yx} &= \frac{1}{32\sqrt{5}} = 0.014
\end{flalign*}
\\
$$
\therefore \frac{\partial^2 f}{\partial xy} = \frac{\partial^2 f}{\partial yx}
$$
\\~\\
\\~\\
Napomena: U ostatku seminara neće se prikazivati $\frac{\partial^2 f}{\partial yx}$ jer je jednaka $\frac{\partial^2 f}{\partial xy}$. No, u Python kôdu nalazi se svaki izračun.

\pagebreak

\section{Programski kôd u Pythonu}
Napomena: sav kôd može se pregledati online u Googleovoj Python bilježnici na ovoj poveznici: 
\newline \href{https://colab.research.google.com/drive/1J-faTA8mjOc46u1WNkfacBiw92qrqLD6}{https://colab.research.google.com/drive/1J-faTA8mjOc46u1WNkfacBiw92qrqLD6}
\\
\newline Ili na Repl.it kao pokretljiva Python skripta, ali treba oko 15 sekundi za pokretanje cijelog programa:
\newline \href{https://replit.com/@marinolinic/Marino-Linic#main.py}{https://replit.com/@marinolinic/Marino-Linic}
\\
\newline GitHub poveznica s Python bilježnicom, skriptom, LaTeX kôdom i seminarom u PDF formatu nalazi se ovdje:
\newline \href{https://github.com/MarinoLinic/matematika-3-seminar}{https://github.com/MarinoLinic/matematika-3-seminar}

\subsection{Biblioteke}
\includegraphics[width=12cm]{Screenshot_1497.png}
\\ 
\\
Uključujemo biblioteku \texttt{sympy}:
\\ - \textbf{sqrt}: ovo je funkcija koja računa korijen.
\\ - \textbf{asin}: ovo je funkcija koja računa arkus sinus.
\\ - \textbf{diff}: ovo je funkcija koja računa derivacije i parcijalne derivacije.
\\ - \textbf{plot3d}: ovo je funkcija koja crta funkciju u trodimenzionalnom prostoru.
\\ - \textbf{symbols}: ovo je funkcija koja definirane simbole uključuje u matematičke operacije.

\pagebreak

\subsection{Računanje parcijalnih derivacija}
\\~\\
\includegraphics[width=12cm]{Screenshot_1498.png}
\\~\\
U nizu sljedećih slika koristimo funkciju \texttt{sympy.diff()} koja nam derivira izraz. Prima dva argumenta: prvi je funkcija, koju smo upisali u varijablu \texttt{f}, a drugi je po kojem simbolu se derivira u slučaju da ima više od jednoga, poput ovoga s parcijalnom derivacijom.
\\~\\
No, prije svega koristimo \texttt{sympy.symbols()} kako bi program znao tretirati slova u izrazu kao nepoznanice. \texttt{sympy.asin()} funkcija odgovara računanju arkusa sinusa, a \texttt{sympy.sqrt()} korijena. 
\\~\\
U kôdu za svaku ćeliju u komentarima opisujemo po čemu se derivira, a naposljetku ubacujemo ime varijable s ciljem da \texttt{sympy} biblioteka prikaže izraz koristeći LaTeX.
\\~\\
Na vrhu ćelije, prva linija govori nam dakle što program radi, odnosno po čemu izraz derivira, a zadnja nam prikazuje rezultat. Računamo parcijalne derivacije prvoga reda po x i y te parcijalne derivacije drugoga reda po x, y i xy.
\\~\\
\includegraphics[width=12cm]{Screenshot_1499.png}
\\~\\
\includegraphics[width=12cm]{Screenshot_1500.png}
\\~\\
\includegraphics[width=12cm]{Screenshot_1501.png}
\\~\\
\includegraphics[width=12cm]{Screenshot_1502.png}
\\~\\
\includegraphics[width=12cm]{Screenshot_1503.png}
\\~\\
\pagebreak
\\
Nakon što smo izračunali parcijalne derivacije, izračunati ćemo ih i u točki T(9,5). Koristimo funkciju \texttt{subs()} kojoj prosljedujemo varijablu tipa \texttt{dict} te slijedno u funkciji \texttt{f(x,y)} uvrštavamo 9 i 5. \textit{Primijetimo da rezultati odgovaraju onima koje smo mi ručno izračunali u prijašnjem dijelu ovoga seminara.}
\\ \\
\includegraphics[width=12cm]{Screenshot_1510.png}
\\
\includegraphics[width=12cm]{Screenshot_1523.png}
\\~\\
Naposlijetku dodajemo printanje svih vrijednosti u konzoli. Koristimo funkciju \texttt{format} kako bi ubacili vrijednosti u prostor vitičastih zagrada. Funkcija prima onoliko argumenata koliko vitičastih zagrada postoji.

\pagebreak

\subsection{Crtanje funkcije i parcijalnih derivacija prvoga reda}
\subsubsection{Funkcija}
Prvo prikazujemo funkciju:
$$
f(x,y) = arcsin \sqrt{\frac{x-y}{x}}
$$
koristeći programsku funkciju \texttt{plot3d()} biblioteke \texttt{sympy}. Prvi argument je funkcija grafa, a drugi argument naslov grafikona. Prije toga smo napravili naslov tipa varijable \texttt{string} spajajući samo funkciju i njezin naziv.
\\~\\
\includegraphics[width=12 cm]{Screenshot_1515.png}

\pagebreak

\subsubsection{Parcijalna derivacija prvog reda po x}
Prikazujemo parcijalnu derivaciju po x (izračun \texttt{sympy} biblioteke):
$$
\frac{\partial f}{\partial x} &= \frac{x\sqrt{\frac{x-y}{x}} \left(\frac{1}{2x} - \frac{x-y}{2x^2}\right)}{\sqrt{1-\frac{x-y}{x}} (x-y)}
$$
odnosno po ručno izračunatom rezultatu:
$$
\frac{\partial f}{\partial x} &= \frac{1}{2x} \sqrt{\frac{y}{x-y}}
$$
\\~\\
\\~\\
\includegraphics[width=12 cm]{Screenshot_1511.png}

\pagebreak

\subsubsection{Parcijalna derivacija prvog reda po y}
Prikazujemo parcijalnu derivaciju po y (izračun \texttt{sympy} biblioteke):
$$
\frac{\partial f}{\partial y} &= -\frac{\sqrt{\frac{x-y}{x}}}{2\sqrt{1-\frac{x-y}{x}} (x-y)}
$$
odnosno po ručno izračunatom rezultatu:
$$
\frac{\partial f}{\partial y} &= \frac{-1}{2\sqrt{y(x-y)}}
$$
\\~\\
\\~\\
\includegraphics[width=12 cm]{Screenshot_1512.png}

\pagebreak

\section{Dodaci}
Sljedeće linije kôda su zakomentirane. Može se reproducirati ove grafikone odkomentiranjem.
\\
\includegraphics[width=10cm]{Screenshot_1519.png}
\\~\\
\includegraphics[width=10cm]{Screenshot_1520.png}
\\~\\
\includegraphics[width=10cm]{Screenshot_1521.png}
\\~\\
\includegraphics[width=10cm]{Screenshot_1522.png}
\\
Ovdje možemo vidjeti da ručno izračunata parcijalna derivacija daje isti grafikon kao i onaj od \texttt{sympy} biblioteke.

\pagebreak

\section{Zaključak}
Spomenut ćemo još jednom dakle rezultate parcijalnih derivacija drugoga reda kao rješenje za zadatak ovog seminara:
\\~\\
\textbf{Parcijalna derivacija drugog reda po x:}
$$
\frac{\partial^2 f}{\partial x^2} &= \frac{2y \sqrt{y} -3x\sqrt{y} }{4x^2(x-y)^{\frac{3}{2}}} 
$$
\\~\\
U točki $T(9, 5)$,
\begin{flalign*}
    \frac{\partial^2 f}{\partial x^2} &=\frac{ -17\sqrt{5} }{2592}
\end{flalign*}
\\~\\
\textbf{Parcijalna derivacija drugog reda po y:}
$$
\frac{\partial^2 f}{\partial y^2} &= \frac{x-2y}{4(y(x-y))^{\frac{3}{2}}}
$$
\\~\\
U točki $T(9, 5)$,
\begin{flalign*}
    \frac{\partial^2 f}{\partial y^2} &= \frac{-1}{160\sqrt{5}}
\end{flalign*}
\\~\\
\textbf{Parcijalna derivacija drugog reda po xy i yx:}
$$
\frac{\partial^2 f}{\partial xy} &= \frac{\partial^2 f}{\partial yx} &= \frac{1}{4\sqrt{y}(x-y)^{\frac{3}{2}}}
$$
\\~\\
U točki $T(9, 5)$,
\begin{flalign*}
    \frac{\partial^2 f}{\partial xy} = \frac{\partial^2 f}{\partial yx} &= \frac{1}{32\sqrt{5}}
\end{flalign*}

\pagebreak

\section{Literatura}
\\~\\
\textbf{1} - ``Derivacija", Wikipedija. \href{https://hr.wikipedia.org/wiki/Derivacija}{https://hr.wikipedia.org/wiki/Derivacija}
\\~\\
\textbf{2} - ``Derivacija funkcije jedne varijable", Marija Maksimović, 2022.
\\~\\
\textbf{3} - ``Funkcije više varijabli", grad.hr. \href{http://www.grad.hr/nastava/matematika/mat2/node3.html}{grad.hr/nastava/matematika/mat2/node3.html}
\\~\\
\textbf{4} - ``Derivacije funkcije više varijabli", Marija Maksimović, 2022.

\end{document}